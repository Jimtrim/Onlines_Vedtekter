\chapter{Generalforsamlingen}

Generalforsamlingen er linjeforeningens øverste organ og er uavhengig av gjeldende hovedstyrevedtak. Generalforsamlingen avholdes årlig i løpet av vårsemesteret. Hoved-styret har ansvar for å samle inn saker som medlemmene ønsker å ha på \mbox{dagsordenen}. \newline

Den ordinære generalforsamlingen skal behandle årsmelding, innsendte saker, \linebreak vedtektsendringer, regnskap og budsjettforslag for neste år. De økonomiansvarlige i linjeforeningen fremlegger reviderte regnskap for generalforsamlingen. Valgkomite må velges for det neste året. \newline

Alle medlemmer i linjeforeningen har stemme- og talerett og har rett til å levere forslag til saker.

%----
\section{Frister}
\label{sec:frister}
\begin{liste}
	\item Innkalling skal sendes ut til medlemmene senest \emph{fire uker} før \mbox{generalforsamlingen} skal avholdes.
	\item Saksforslag og forslag til vedtektsendringer sendes Hovedstyret senest \emph{to uker} før generalforsamlingen skal avholdes.
	\item Fullstendig saksliste med vedtektsendringer skal tilgjengeliggjøres senest \emph{en uke} før møtedato. Denne skal også inneholde årsmelding, revidert regnskap, budsjettforslag og eventuelle andre relevante sakspapirer.
	\item Referat fra generalforsamlingen skal underskrives av paraferer og sendes \linebreak medlemmene eller gjøres tilgjengelig for medlemmene senest 14 dager etter generalforsamlingen.
\end{liste}


%----
\section{Ekstraordinær generalforsamling}
Denne kan innkalles av Hovedstyret eller om minst 1/8 av medlemmene ønsker det. Fristene for å kalle inn til ekstraordinær er halvert i forhold til fristene for ordinær, jamfør §3.1. 

%----
\section{Organisering}
\label{sec:organisering}
Ved generalforsamling er disse vervene nødvendig: 

\begin{liste}
	\item Ordstyrer
	\item To referenter - skriver referat under generalforsamling og samarbeider om \mbox{renskriving}
	\item Tellekorps\space \space \space \space - teller opp stemmer ved avstemming
	\item To paraferer - godkjenner referat fra generalforsamling
	
\end{liste}

%----
\newpage
\section{Beslutningsdyktighet og avstemming}

For at en generalforsamling skal være beslutningsdyktig må et visst antall medlemmer ha møtt opp. Antallet defineres som det laveste mellom 15 medlemmer og 1/5 av medlemmene. Alle saker på generalforsamlingen avgjøres ved alminnelig flertall, med unntak av vedtektsendringer som avgjøres med kvalifisert flertall.

\begin{liste}
	\item Alminnelig flertall er definert som 1/2 av de oppmøtte med stemmerett.
	\item Kvalifisert flertall er definert som 2/3 av de oppmøtte med stemmerett.
\end{liste}

Verken forhåndsstemming eller fullmakter er tillatt å bruke ved avstemming.

%----

\section{Gjennomføring av valg}{
Dersom det er mer enn en kandidat til et verv skal det avholdes anonymt valg for det aktuelle vervet. Ved
avholdt valg vil alle stemmer, inkludert blanke, være \linebreak tellende. For å regnes som vinner av valget må en kandidat få over halvparten (1/2) av stemmene. Ved manglende flertall fjernes kandidaten med færrest stemmer \linebreak og valget går inn i en ny runde. \newline

Ved manglende flertall på kandidat med flest stemmer og stemmelikhet på de med færrest stemmer vil det
avholdes en fullstendig ny runde. \newline

Innehavere av verv sitter inntil det er gjennomført et godkjent valg for det respektive vervet. Dersom
generalforsamlingen ikke klarer å gjennomføre et suksessfullt valg må det kalles inn til ekstraordinær
forsamling innen tre dager etter endt ordinær forsamling.
}
