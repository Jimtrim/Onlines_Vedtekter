\chapter{Medlemskap}
\label{chap:medlemskap}
Studenter ved følgende studier kan bli medlem av linjeforeningen: 
\begin{liste}
	\item Årstudium i informatikk (ÅIT)
	\item Bachelor i informatikk (BIT)
	\item Master i informatikk (MIT)
	\item Internasjonale studenter innen Information Systems
	\item Erfaringsbasert master i helseinformatikk		% Lagt til etter genfors 2012
	\item Bioinformatics (internasjonal master)		% Lagt til etter genfors 2012
	\item Healthcare informatics (internasjonal master)	% Lagt til etter genfors 2012
	\item Lektorutdanning i realfag med informatikk som studieretning % Lagt til etter genfors 2013
	\item Masterstudenter på Entreprenørskolen som tok bachelor i informatikk % Lagt til etter genfors 2014
\end{liste} 
Følgende er medlemmer av linjeforeningen på livstid:
\begin{liste}
	\item Æresmedlemmer
	\item Medlemmer av pensjonistkomiteen
	\item Medlemmer av Eldsterådet
\end{liste}

Et medlem som har deltatt på linjeforeningsopptaket Kompileringen betegnes som et kompilert program. Et medlem som ikke har deltatt på opptaket betegnes som et tolket program. Etter fullført opptak vil et medlem motta sin kompileringspin. Mistet pin kan erstattes ved at medlemmet kjøper ny. Kun personer som har gjennomført opptaket har tillatelse til å bære kompileringspinen. Med unntak av retten til å bære kompileringspin har alle medlemmer like rettigheter. % Lagt til etter genfors 2012

%----
\section{Medlemmers rettigheter}

Medlemmer av linjeforeningen kan kreve innsyn i linjeforeningens regnskap og kan få innsyn i ikke-konfidensielle vedtak. Medlemmer som har utført et arbeid for linjeforeningen har rett til å få attest av Hovedstyret ved forespørsel.

%----
\section{Æresmedlemskap}

Linjeforeningen har anledning til å utnevne æresmedlemmer. Æresmedlemskap går \linebreak til personer som har gjort en eksepsjonell innsats for informatikkens sak, eller har gjort en eksepsjonell innsats for studentene ved informatikkstudiet. \newline 


Utnevnelse skjer ved at Hovedstyret mottar forslag fra medlemmer av linjeforeningen, med begrunnelse om hvorfor kandidaten bør utnevnes. Kandidaten utnevnes ved kvalifisert flertall i Hovedstyret.

%----
\section{Utmelding, opphør, ekskludering}
Alle medlemmer kan melde seg ut av linjeforeningen ved å melde fra skriftlig til \mbox{Hovedstyret}. Utmelding gjelder fra utmeldingsdato. Når en person ikke lenger \linebreak oppfyller kravene for medlemskap i linjeforeningen vil medlemskapet opphøre. \newline

Ved grov uaktsomhet eller brudd på norsk lov kan Hovedstyret, med kvalifisert \linebreak flertall, fatte vedtak om å eksludere et medlem ut av linjeforeningen. \newline

%----
\section{Adgang for andre til å bli medlem}

På spesielt grunnlag kan Hovedstyret, ved kvalifisert flertall, tillate andre å bli medlem. Dette gjelder primært studenter ved andre fakulteter og linjer som har informatikk som et av sine hovedstudier. 