\chapter{Struktur, ledelse og organisasjon}
\section{Hovedstyret}
Hovedstyret er linjeforeningens høyeste organ mellom generalforsamlingene. Hovedstyrets medlemmer velges på generalforsamlingen og skal drive linjeforeningen mellom generalforsamlingene. For at Hovedstyret skal være beslutningsdyktig må minst fire representanter være tilstede. \newline

Ingen kan inneha to verv i Hovedstyret. Leder har dobbeltstemme ved stemmelikhet. Hovedstyrets møter er lukket, men gjester kan inviteres dersom Hovedstyret ønsker \linebreak dette. Leder av Hovedstyret skal avholde medarbeidersamtaler for hovedstyre-\linebreak medlemmer minst en gang i året. Hovedstyret skrives med stor 'H' på samme måte som egennavn.

%----
\under{Hovedstyrets sammensetning}{
Hovedstyret består av:
\begin{liste}
	\item Leder
	\item Nestleder
	\item Ledere for de komiteene som er beskrevet under §\ref{sec:komiteer}, bortsett fra leder for pensjonistkomiteen.
\end{liste}
}

%----
\under{Krav til kandidater}{
En kandidat må ha innehatt et verv i linjeforeningen i minst ett semester med unntak av kandidatur til stillingen som leder av bank- og økonomikomiteen.
}

%----
\under{Hovedstyrets virke}{

\begin{liste}
	\item Hovedstyret fører logg over linjeforeningens aktiviteter og fremlegger årsberetning for generalforsamlingen.
	\item Hovedstyret skal formidle saker og vedtak til komitemedlemmene via den respektive komitelederen.
	\item Representantene av Hovedstyret kan uttale seg og handle på linjeforeningens vegne når vedkommendes sak er avklart i Hovedstyret. Ved uklarheter skal leder uttale seg.
	\item Hovedstyret har fullmakt til å fylle eventuelle tomme verv ved behov.
	\item Hovedstyret er ansvarlig for komiteopptak og skal føre intervju- og søknadsprosesser. 			Hovedstyret skal godkjenne komiteenes retningslinjer på første hovedstyremøte etter endt generalforsamling.
\end{liste}
}



%----
\vspace{-23pt}
\section{Komiteer}
\label{sec:komiteer}

Alle komiteer består av minimum en leder og en økonomiansvarlig, hvorav det er valgfritt om økonomiansvarlig er Leder av banKom eller en som sitter i den \linebreak respektive komiteen. Økonomiansvarlig skal holde orden på økonomien \mbox{i tråd med §\ref{chap:okonomi}}. Enhver komite skal utarbeide sine egne retningslinjer som skal legges frem for og godkjennes av Hovedstyret. \newline

Leder av en komite har mulighet til å opprette egne komitestillinger, for eksempel nestleder, og det er opp til komiteen selv å fylle stillingen. Enhver leder skal avholde medarbeidersamtaler minst en gang i året. \newline

Bare medlemmer av linjeforeningen kan inneha verv og disse opphører dersom studenten bytter til studium som ikke kvalifiserer til medlemskap i linjeforeningen \mbox{etter §\ref{chap:medlemskap}.} \newline

%----
\vspace{-10pt}
\under{Arrangementskomiteen}{
Komiteens hovedoppgave er å koordinere og gjennomføre sosiale arrangement. Komiteens navn forkortes arrKom.
}

%----
\vspace{-10pt}
\under{Bank- og økonomikomiteen}{
Komiteens hovedoppgave er administrere linjeforeningens økonomi. Komiteens  medlemmer utgjøres av de økonomiansvarlige fra de andre komiteene. Komiteens navn forkortes banKom.
}

%----
\vspace{-10pt}
\under{Bedriftskomiteen}{
Komiteens hovedoppgaver er å være et bindeledd mellom linjeforeningens medlemmer og næringslivet, og å utarbeide en hovedsamarbeidssamtale med en bedrift for Linjeforeningen Online. Komiteens navn forkortes bedKom.
}

%----
\vspace{-10pt}
\under{Drifts- og utviklingskomiteen}{
Komiteens hovedoppgave er å utvikle og vedlikeholde linjeforeningens datasystemer. Komiteens navn forkortes dotKom.
}

%----
\vspace{-10pt}
\under{Fag- og kurskomiteen}{
Komiteens hovedoppgave er å koordinere og gjennomføre arrangement som tilbyr faglig innhold, primært for linjeforeningens egne medlemmer. Komiteens navn forkortes fagKom.
}

%----
\vspace{-10pt}
\under{Pensjonistkomiteen}{
Medlemskap i pensjonistkomiteen kan søkes til etter avsluttet studie på informatikk ved NTNU.\newline

Et medlemsskap i pensjonistkomiteen varer livet ut med mindre man på nytt blir medlem av en av linjeforeningens komiteer; man vil da få permisjon fra pensjonistkomiteen i den tidsperioden man er med i
den andre komiteen. Alle tidligere informatikkstudenter har anledning til å søke om medlemskap i pensjonistkomiteen.\newline

Det påpekes at epostlistene til pensjonistkomiteen ikke burde benytte seg av epostadresser fra universitetet, det bør heller benyttes en permanent, privat epostadresse.\newline

Komiteens navn forkortes pangKom.
}

%----
\vspace{-10pt}
\under{Profil- og aviskomiteen}{
Komiteens hovedoppgave er å sikre kvalitet på profileringsmateriell, samt gi ut
linjeforeningens tidsskrift. Komiteens navn forkortes proKom.
}

%----
\vspace{-10pt}
\under{Trivselskomiteen}{
Komiteens hovedoppgave er å sørge for økt trivsel blant informatikere i hverdagen. Komiteens sekundære oppgave er å ha ansvaret for linjeforeningskontoret. Ansvaret for kontoret innebærer å planlegge og følge opp kontorvakter, tilrettelegge for møteaktivitet og sørge for at komitemedlemmer har tilgang til kontoret. Komiteens navn forkortes triKom.
}

%----
\vspace{-10pt}
\under{Seniorkomiteen}{
Komiteens hovedoppgave vil være å bistå med kunnskap, erfaring og assistanse i Linjeforeningens daglige drift. For å søke seg til Seniorkomiteen må man ha hatt et aktivt verv i Linjeforeningen i minst fire semester. Seniorkomiteen tar selv opp medlemmer, men medlemmene må godkjennes av Hovedstyret. Dersom Seniorkomiteen ikke har mulighet til å gjennomføre opptaket vil Hovedstyret utføre denne oppgaven. \newline

Seniorkomiteen velger selv sin leder. Leder av Seniorkomiteen har møte- og talerett i Hovedstyret.
}

%----
\vspace{-10pt}
\section{Nodekomiteer}{
En nodekomite er underlagt en av komiteene beskrevet i §4.2. Komiteen som nodekomiteen er underlagt kalles heretter forelderkomiteen.

Forelderkomiteen nedsetter en leder for nodekomiteen ved opprettelse. Ved tillatelse fra forelderkomiteen kan nodekomiteen i stedet velge sin egen leder etter at medlemmer av nodekomiteen er tatt opp. Denne lederen er ansvarlig for å ta opp nødvendige medlemmer, opprette nødvendige stillinger og holde forelderkomiteen løpende underrettet om komiteens status. Leder for forelderkomiteen er i sin tur ansvarlig for å holde Hovedstyret løpende underrettet om nodekomiteens status.

I likhet med komiteene under §4.2 skal en nodekomite ha egen økonomiansvarlig. I en nodekomite der Hovedstyret stiller med midler skal budsjett godkjennes av Hovedstyret. Regnskap skal uansett fremlegges på generalforsamlingen.

Retningslinjer for en nodekomite skal fremlegges for, og godkjennes av, Hovedstyret etter generalforsamlingen på lik linje med komiteene under §4.2.
}

%----
\under{Periodiske komiteer}{
Dette er nodekomiteer som eksisterer i kortere perioder, men med faste intervall, for å ta seg av spesielle begivenheter. Retningslinjer for periodiske komiteer skal fremlegges for, og godkjennes av, Hovedstyret snarlig etter opprettelse av komiteen i stedet for etter generalforsamlingen.
}

%----
\underunder{Jubileumskomiteen}{
Komiteens hovedoppgave er å organisere arrangement i forbindelse med linjeforeningens jubileer. Komiteens forelderkomite er arrKom. Komiteens navn forkortes jubKom. \newline

Hovedstyret skal spare kr. 10.000,- hvert år til neste jubileum. Beløpet kan justeres av Hovedstyret dersom de ser behov for det. Hovedstyret kan etter behov vedta å ikke spare noe et år dersom linjeforeningen har havnet i en vanskelig økonomisk situasjon. Pengene skal settes over på en egen konto og blir da øremerket til gjennomføring av linjeforeningens neste jubileum.
}

%----
\underunder{Velkomstkomiteen}{
Komiteens hovedoppgave er å organisere fadderperiode for nye studenter som oppfyller de krav for medlemskap som er listet under §\ref{chap:medlemskap}. Komiteens navn forkortes velKom.
}

%----
\under{Krysskomiteer}{
Dette er nodekomiteer som har medlemmer utenfor linjeforeningen. Krysskomiteer kan operere som frittstående organisasjoner, men vil likevel få tilgang til tjenestene linjeforeningens komiteer tilbyr. Det bemerkes at det som beskrives her kun gjelder komitemedlemmer i en krysskomite som også er medlemmer av linjeforeningen. Krysskomiteer kan selv bestemme om de ønsker å ha retningslinjer slik som de andre komiteene.
}

%----
\underunder{Casual Gaming}{
Komiteens hovedoppgave er å organisere LAN. Komiteens forelderkomite er arrKom.
}

%----
\under{Redaksjonen}{
Komiteens hovedoppgave er å gi ut linjeforeningens avis. Redaktøren står fritt fra linjeforeningen, men er underlagt de retningslinjer og avtaler som finnes mellom linjeforeningen og forelderkomiteen. Tidsskriftet skal, så langt det lar seg gjøre, følge Vær Varsom-plakaten definert av Pressens Faglige Utvalg. Redaktøren følger på lik måte Redaktørplakaten definert av Pressens Faglige Utvalg. Redaktøren står fritt til å velge redaksjonsmedlemmer, også blant personer utenfor linjeforeningen. Merk at personer som ikke innfrir krav til medlemskap som definert under §5 ikke får medlemskap i linjeforeningen utenfor redaksjonen, men kan fritt inkluderes på interne arrangementer for medlemmer med verv. Komiteens forelderkomite er proKom.
}

\section{Permisjon eller oppsigelse fra komite}{

Ved permisjon fra en komite er man fullstendig fritatt de pliktene komitevervet medførte. Man er i tillegg unntatt visse administrative rettigheter som bestemmes av den respektive komiteen. \newline

Ved oppsigelse fra en komite mister man de rettigheter man har tilegnet seg internt i komiteen. Man har likevel rett til å bruke de av linjeforeningens og komiteens effekter man har fortjent, samt å bli med i pangKom dersom man møter kravene for opptak.
}

%----
\under{Pause i sitt engasjement}{
Et komitemedlem kan søke om permisjon fra den respektive komite når medlemmet ønsker å ta en pause fra komiteen. Man må ha \mbox{sittet} i en komite i minst ett semester for å kunne søke permisjon. Dersom \linebreak permisjonen varer lengre enn to semestere vil medlemmets verv opphøre. Permisjonslengde kan ikke overstige komitemedlemmets fartstid i den \linebreak respektive komiteen.
}

\under{Verv i Hovedstyret}{
Dersom et komitemedlem blir valgt til et av følgende hovedstyreverv vil medlemmet automatisk få permisjon fra sin komite, og kan fritt \linebreak returnere til denne ved endt engasjement i Hovedstyret:
\begin{liste}
	\item Leder
	\item Nestleder
	\item Leder for banKom
\end{liste}
}

\under{Advarsel og oppsigelse}{
Leder av en komite har rett til å si opp et medlem av sin egen komite. Oppsigelse skal kun finne sted i tilfeller der det blir ansett som høyst nødvendig for å beskytte komiteens samhold, initiativ, integritet eller profesjonalitet. Et komitemedlem har rett til å få én advarsel og en mulighet til å forbedre seg i forkant av en eventuell oppsigelse. Leder av komiteen plikter å konsultere leder av linjeforeningen i forkant av en eventuell advarsel eller oppsigelse. Det påpekes at leder av linjeforeningen har HMS-ansvar.
}

\section{Kontoret}

Kontorets retningslinjer defineres av trivselskomiteen. Kontorarealet skal \linebreak primært brukes til møtevirksomhet for komiteene og Hovedstyret. Dersom kontoret ikke er opptatt i forbindelse med møtevirksomhet skal det etterstrebes å holde \linebreak kontoret åpent for sosialt samvær og andre ærend medlemmer måtte ha som \linebreak fordrer personlig oppmøte.


\newpage
\section{Eldsterådet}
Eldsterådet er en gruppe bestående av særs erfarne medlemmer som har opparbeidet seg status og autoritet
blant medlemmene i linjeforeningen. \newline

Eldsterådet skrives med stor 'E' på samme måte som egennavn.

\under{Medlemskap}{
Medlemmer av linjeforeningen blir medlem i Eldsterådet ved å oppfylle en eller flere av følgende krav:
\begin{liste}
	\item Er eller har vært leder av linjeforeningen.
	\item Er eller har vært et aktivt hovedstyremedlem i tre år (seks semestere) eller mer.
	\item Har blitt utnevnt til Æresmedlem.
\end{liste}
Det spesifiseres at man kan være medlem i Eldsterådet samtidig som man er medlem av Hovedstyret eller en
av linjeforeningens komiteer.
}

\under{Formål} {
Det er Eldsterådets oppgave å avholde linjeforeningsopptaket (ikke komiteopptak). Før opptaket skal
sittende leder av linjeforeningen \linebreak velge representanter fra Eldsterådet til å avholde opptaket sammen med lederen. Det bør være minst fem medlemmer av Eldsterådet til stede \linebreak under opptaket. Det spesifiseres at leder bør invitere til dette i god tid før linjeforeningsopptaket da medlemmer av Eldsterådet gjerne bor langt vekk fra universitetet.
}

\section{Mislighold av verv}
Om et komitemedlem eller en innehaver av linjeforeningsverv misligholder sine arbeidsoppgaver, kan ethvert medlem av linjeforeningen stille mistillitsforslag ovenfor vedkommende. Mistillitsforslaget skal leveres skriftlig til Hovedstyret, som skal behandle saken. Ved mistillitsforslag mot et hovedstyremedlem blir den anklagede suspendert inntil Hovedstyret har kommet med en avgjørelse. Mistillitsforslaget leses opp i Hovedstyret, deretter skal den anklagede få en mulighet til å forsvare seg før Hovedstyret diskuterer og avgjør saken uten den anklagede til stede. For å beholde et beslutningsdyktig og fungerende Hovedstyre vil det kun være mulig å stille mistillitsforslag mot ett hovedstyremedlem av gangen. Hovedstyret har to uker på å behandle et mistillitsforslag.


\section{Vervvarighet}
Et verv i Linjeforeningen varer i tre år fra måneden man ble tatt opp. Dersom man ønsker å være et aktivt komitemedlem etter disse tre årene kan man søke til Hovedstyret om forlengelse av Online-vervet for ett år av gangen. Alle verv i Linjeforeningen teller på de tre vervårene, inkludert verv i Hovedstyret, men ekskludert verv i Seniorkomiteen

%--------------------------------------------
\newpage
